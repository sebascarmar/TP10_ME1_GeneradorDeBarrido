  \pagebreak
  \section{Actividad Práctica}
    Se propone medir la respuesta en frecuencia de un receptor FM mediante un generador de barrido y marcas, junto
    con un osciloscopio analógico. Los instrumentos de los cuales se hace uso son

    \begin{itemize}
      \item Generador de barrido y marcas LSW-250
      \item Osciloscopio analógico Hitachi V-665A
      \item Radio FM
    \end{itemize}
    
    El esquema de conexiones que se debe implementar con los dispositivos antes mencionados, se puede observar en
    la Figura~\ref{fig:EsquemaConexiones}.

    \begin{figure}[H]
      \centering
      \frame{\includegraphics[width=0.6\textwidth]{Imagenes/ActividadPractica/EsquemaConexiones.png}}
      \caption{Esquema de conexiones para las mediciones.}
      \label{fig:EsquemaConexiones}
    \end{figure}
    
    
      \subsection{Calibración del dial del Generador de Barrido}
    A 

      \subsection{Características de detección}

    El esquema de conexiones que se debe implementar con los dispositivos antes mencionados, se puede observar en
    la Figura~\ref{fig:EsquemaConexiones}. 

    \begin{figure}[H]
      \centering
      \frame{\includegraphics[width=0.6\textwidth]{Imagenes/ActividadPractica/CaracteristicasDeDeteccion/EsquemaConexiones.png}}
      \caption{Esquema de conexiones para las mediciones.}
      \label{fig:EsquemaConexiones}
    \end{figure}

    Luego, en la Figura~\ref{fig:InstrumentosParaFI} se encuentra la implementación del experimento. Además, a 
    modo de apreciar el funcionamiento del generador de barrido y marcas, en la Figura~\ref{fig:EspectroaFI} se 
    logra ver la respuesta en frecuencia junto con la marca.

    \begin{figure}[H]
      \centering
      \begin{subfigure}[ht]{0.48\textwidth}
        \frame{\includegraphics[width=\textwidth]{Imagenes/ActividadPractica/CaracteristicasDeDeteccion/Exp2.1_AmpliFI_InstrumentosConMarcaAIzquierda.jpg}}
        \caption{Disposición de instrumentos.}
        \label{fig:InstrumentosParaFI}
      \end{subfigure}
      \hfill 
      \begin{subfigure}[ht]{0.48\textwidth}
        \frame{\includegraphics[width=\textwidth]{Imagenes/ActividadPractica/CaracteristicasDeDeteccion/Exp2.2_AmpliFI_MarcaAIzquierda.jpg}}
        \caption{Seteo de la espectro junto con la marca.}
        \label{fig:EspectroaFI}
      \end{subfigure}

      \caption{Espectro del amplificador de FI del detector.}
      \label{fig:MediccionEspectroFIDetector}
    \end{figure}

    Ahora, se procede a la medición de la frecuencia mínima del conjunto del detector y el amplificador de FI,
    cuyo valor es $\mathbf{f_{FImin} = 10,25~MHz}$, tal y como se observa en la Figura~\ref{fig:FrecuenciaMinFI}.

    \begin{figure}[H]
      \centering
      \begin{subfigure}[ht]{0.48\textwidth}
        \frame{\includegraphics[width=\textwidth]{Imagenes/ActividadPractica/CaracteristicasDeDeteccion/Exp2.4_AmpliFI_FcMin.jpg}}
        \caption{Marca en la frecuencia mínima.}
        \label{fig:FrecuenciaMinFI_Osc}
      \end{subfigure}
      \hfill 
      \begin{subfigure}[ht]{0.48\textwidth}
        \frame{\includegraphics[width=\textwidth]{Imagenes/ActividadPractica/CaracteristicasDeDeteccion/Exp2.5_AmpliFI_FcMin_DialMarca.jpg}}
        \caption{Medición de frecuencia mínima.}
        \label{fig:FrecuneciaMinFI_Gener}
      \end{subfigure}

      \caption{Medición de frecuencia mínima del amplificador de FI y el detector.}
      \label{fig:FrecuenciaMinFI}
    \end{figure}

    De la misma forma, se lleva la marca a la posición central del espectro, y se mide la frecuencia central 
    del detector y el amplificador de FI. En la Figura~\ref{fig:FrecuenciaCenFI} se puede ver que dicha 
    frecuencia medida es $\mathbf{f_{FIcen} = 10,45~MHz}$.

    \begin{figure}[H]
      \centering
      \begin{subfigure}[ht]{0.48\textwidth}
        \frame{\includegraphics[width=\textwidth]{Imagenes/ActividadPractica/CaracteristicasDeDeteccion/Exp2.9_AmpliFI_FCentral.jpg}}
        \caption{Marca en la frecuencia central.}
        \label{fig:FrecuenciaCenFI_Osc}
      \end{subfigure}
      \hfill 
      \begin{subfigure}[ht]{0.48\textwidth}
        \frame{\includegraphics[width=\textwidth]{Imagenes/ActividadPractica/CaracteristicasDeDeteccion/Exp2.9_AmpliFI_FCentral_DialMarca.jpg}}
        \caption{Medición de frecuencia central.}
        \label{fig:FrecuneciaCenFI_Gener}
      \end{subfigure}

      \caption{Medición de frecuencia central del amplificador de FI y el detector.}
      \label{fig:FrecuenciaCenFI}
    \end{figure}

    Por último, la frecuencia máxima en cuestión se puede apreciar en la Figura~\ref{fig:FrecuenciaMaxFI}, 
    la cual da un valor de $\mathbf{f_{FImax} = 10,65~MHz}$.

    \begin{figure}[H]
      \centering
      \begin{subfigure}[ht]{0.48\textwidth}
        \frame{\includegraphics[width=\textwidth]{Imagenes/ActividadPractica/CaracteristicasDeDeteccion/Exp2.6_AmpliFI_FcMax.jpg}}
        \caption{Marca en la frecuencia máxima.}
        \label{fig:FrecuenciaMaxFI_Osc}
      \end{subfigure}
      \hfill 
      \begin{subfigure}[ht]{0.48\textwidth}
        \frame{\includegraphics[width=\textwidth]{Imagenes/ActividadPractica/CaracteristicasDeDeteccion/Exp2.7_AmpliFI_FcMax_DialMarca.jpg}}
        \caption{Medición de frecuencia máxima.}
        \label{fig:FrecuneciaMaxFI_Gener}
      \end{subfigure}

      \caption{Medición de frecuencia máxima del amplificador de FI y el detector.}
      \label{fig:FrecuenciaMaxFI}
    \end{figure}

    Los valores obtenidos durante esta experiencia se encuentran en forma tabulada en la Tabla~\ref{tab:MedicionesFI}.

        \begin{table}[H]
            \small
            \centering
            \begin{tabular}{cccc}
                 \toprule
                 \textbf{Frec. central} $\mathbf{[MHz]}$ &  \textbf{Frec. mín.} $\mathbf{[MHz]}$ & 
                  \textbf{Frec. máx.} $\mathbf{[MHz]}$ &  $\mathbf{\triangle}$ \textbf{f} $\mathbf{[MHz]}$  \\ \midrule
                 10,45   &   10,25   & 10,65   & 0,4 \\ 
                 \bottomrule
            \end{tabular}
            \caption{Mediciones de FI obtenidas.}
            \label{tab:MedicionesFI}
        \end{table}

      \subsection{Valores límtes de detección de sintonía}

