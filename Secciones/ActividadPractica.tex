  \pagebreak
  \section{Actividad Práctica}
    Se propone medir la respuesta en frecuencia de un receptor FM mediante un generador de barrido y marcas, junto
    con un osciloscopio analógico. Los instrumentos de los cuales se hace uso son

    \begin{itemize}
      \item Generador de barrido y marcas LSW-250
      \item Osciloscopio analógico Hitachi V-665A
      \item Radio FM
    \end{itemize}

    Para el receptor de FM la etapa detectora se considera como parte del amplificador de FI (frecuencia intermedia).
    
      \subsection{Calibración del dial del Generador de Barrido}
    A 

      \subsection{Características de detección}

      \subsection{Valores límtes de detección de sintonía}

