  \section{Conclusiones}
    Partiendo del diagrama del receptor que se encuentra en la Figura~\ref{fig:ReceptorSimplificado},
    la inversión que se produce en los espectros obtenidos cuando se trabaja con frecuencias de
    señales FM y  FI, se justifica debido a que cuando se inyecta esta última, se está
    salteando la etapa de conversión, y va directo a la etapa de, justamente, el amplificador de FI.
    Por otro lado, una señal de FM sí pasa por la etapa de conversión, la cual produce una convolución 
    en frecuencia, que a su vez conlleva a una inversión del espectro.

    Para determinar la frecuencia del oscilador local, se parte sabiendo las frecuencias 
    \(f_{min}\) y \(f_{max}\) del receptor, y
    aplicando el cálculo 
      \begin{equation*}
        FI = \dfrac{f_{max} - f_{min}}{2} + k = 
        \dfrac{108 - 88~[MHz]}{2} + 0,7~MHz \hspace{20pt} \therefore \hspace{20pt} FI = 10,7~[MHz]~,
      \end{equation*}
    donde \(k = 0,7~MHz\) para el estándar de FM.
    
    Luego, debido a que el receptor es superheterodino, la frecuencia del oscilador local 
    se determina de la siguiente manera
      \begin{equation*}
        f_{osc} = f_{sintonia} +FI~.
      \end{equation*}

    Si se analizara por separado la respuesta en frecuencia del amplificador de \(FI\), lo que
    se obtendría sería un solo pulso, como se muestra a la izquierda de la Figura~\ref{fig:FdeTReceptor}.
    Dicha respuesta estará centrada en \(10,7~MHz\) aproximadamente, y debido a que 
    posee una desviación de frecuencia de \(\triangle f = 0,4~MHz\), el espectro que
    se podría llegar a visualizar estaría bastante acotado, lo cual es propio de la 
    selectividad que posee el amplificador en cuestión.   
    
    La curva de respuesta de un detector de FM, o conversor de frecuencia a tensión, 
    posee dicha forma debido a que, para valores de frecuencia de corte mínima el 
    conversor entrega la más negativa, y para valores  de frecuencia de 
    corte máxima la tensión es la más positiva.   
    
    Para la primer experiencia se vio dos de las señales de salida que proporciona el generador de barrido
    y marcas. Una de ellas proviene de la salida \textit{H}, la cual se trata de una señal triangular que es
    utilizada para la generación del chirp, y además, para generar el eje horizontal del espectro en frecuencias.
    La otra señal proporcionada por la salida  \textit{SWEEP OUT} es la señal modulada en frecuencia, la cual es conectada
    a la red bajo análisis y genera el barrido en frecuencia. Dicha señal se habilita con la pendiente positiva
    de la señal triangular antes mencionada, y se inhabilita con la pendiente negativa. Esto tiene como objetivo
    realizar el barrido en un solo sentido y evitar efectos de histéresis que dificultarían la medición.

