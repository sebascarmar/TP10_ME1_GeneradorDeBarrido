  \section{Conclusiones}

    Para determinar la frecuencia del oscilador local, se parte sabiendo las frecuencias 
    \(f_{min}\) y \(f_{max}\) del receptor, y
    aplicando el cálculo 
      \begin{equation*}
        FI = \dfrac{f_{max} - f_{min}}{2} + k = 
        \dfrac{(108 - 88)~MHz}{2} + 0,7 \hspace{20pt} \therefore \hspace{20pt} FI = 10,7~[MHz]~,
      \end{equation*}
    donde \(k = 0,7\) para el estándar de FM.
    
    Luego, debido a que el receptor es superheterodino, la frecuencia del oscilador local 
    se determina de la siguiente manera
      \begin{equation*}
        f_{osc} = f_{sintonia} +FI~.
      \end{equation*}

    Si se analizara por separado la respuesta en frecuencia del amplificador \(FI\), lo que
    se obtendría seria un solo pulso como se muestra a la izquierda de la Figura 
    \ref{fig:FdeTReceptor}, la cual estará centrada en \(10,7 MHz\) y debido a que 
    posee una desviación de frecuencia de (\(\triangle f = 0,4 MHz\), el espectro que
    se podría llegar a visualizar estaría bastante acotado, esto es propio de la 
    selectividad que posee el filtro de \(FI\).   
    
    La curva de respuesta de un detector de FM o conversor de frecuencia a tensión 
    posee dicha forma debido a que, para valores de frecuencia baja de corte el 
    conversor entrega la mínima tensión y para valores elevados de frecuencia de 
    corte la tensión es la mas alta.   
    
    El generador de barrio y marcas proporciona 2 señales de salidas que poseen ciertas
    características. la primera señal proviene de la salida \textbf{H}, la cual provee 
    una señal triangula que es utilizada para el eje horizontal del espectro en 
    frecuencia, dicha señal es la que se emplea en el canal 1 del osciloscopio. 

    la otra señal proporcionada por la salida  \textbf{SWET OUT} es la señal modulada en 
    frecuencia (chirp), la cual se conecta al canal 2 del osciloscopio y dicha señal
    genera el barrio de frecuencia del eje vertical. La misma posee una pendiente de 
    subida que concuerda con la de la señal del eje horizontal, ademas solo estará 
    habilitada en su flanco de subida para evitar histéresis (o efecto de doble trazo) 
    a la hora de ser visualizada en el osciloscopio.   
